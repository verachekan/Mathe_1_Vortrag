%% LaTeX-Beamer template for KIT design
%% by Erik Burger, Christian Hammer
%% title picture by Klaus Krogmann
%%
%% version 2.1
%%
%% mostly compatible to KIT corporate design v2.0
%% http://intranet.kit.edu/gestaltungsrichtlinien.php
%%
%% Problems, bugs and comments to
%% burger@kit.edu

\documentclass[18pt]{beamer}

%% SLIDE FORMAT

% use 'beamerthemekit' for standard 4:3 ratio
% for widescreen slides (16:9), use 'beamerthemekitwide'
\usepackage{listings}
\usepackage{templates/beamerthemekit}
% \usepackage{templates/beamerthemekitwide}

%% TITLE PICTURE

% if a custom picture is to be used on the title page, copy it into the 'logos'
% directory, in the line below, replace 'mypicture' with the 
% filename (without extension) and uncomment the following line
% (picture proportions: 63 : 20 for standard, 169 : 40 for wide
% *.eps format if you use latex+dvips+ps2pdf, 
% *.jpg/*.png/*.pdf if you use pdflatex)

\titleimage{logologo}

%% TITLE LOGO

% for a custom logo on the front page, copy your file into the 'logos'
% directory, insert the filename in the line below and uncomment it

%\titlelogo{mylogo}

% (*.eps format if you use latex+dvips+ps2pdf,
% *.jpg/*.png/*.pdf if you use pdflatex)

%% TikZ INTEGRATION

% use these packages for PCM symbols and UML classes
% \usepackage{templates/tikzkit}
% \usepackage{templates/tikzuml}

% the presentation starts here

\title[Short title]{Mathe-1}
%\subtitle{Something for XYZ 2009}
\author{Charlotte P., Lena W., Vera C., Christian K.}

\institute{ITI Wagner \& IPD Tichy}

% Bibliography

\usepackage[citestyle=authoryear,bibstyle=numeric,hyperref,backend=biber]{biblatex}
\addbibresource{templates/example.bib}
\bibhang1em

\begin{document}

% change the following line to "ngerman" for German style date and logos
\selectlanguage{ngerman}

%title page
\begin{frame}
\titlepage
\end{frame}

%table of contents
\begin{frame}{Gliederung}
\tableofcontents
\end{frame}

\section {Big Integer}
\begin{frame}{Big integer}
\begin {itemize}
\item die maximale Zahl ist gr"o"ser als integer?
\pause 
\item nehme long long
\pause
\item die Zahl ist gr"o"ser als long long
\pause 
\item ?????????????????????????????????? (Panik) 
\end {itemize}
\end{frame}

\begin{frame}{Big integer - Java nutzen}
\begin {itemize}
\item import java.math.BigInteger
\item Konstruktor: BigInteger(String val)
\item Methoden:
\begin {itemize}
\item BigInteger add(BigInteger val)
\item BigInteger multiply(BigInteger val)
\item BigInteger subtract(BigInteger val)
\item ...
\end {itemize}
\end {itemize}
\end{frame}

\begin{frame} {Laufzeiten}
\begin {itemize}
\item Addition, Subtraktion in $\mathcal{O}(n)$
\item Multiplikation in $\Theta(n^{log_{2}3})$ (Karatsuba)
\end {itemize}
\end{frame}


\begin{frame} {C++? Selbst implementieren!}
\begin {itemize}
\item Addition: Die Tafel ist da $\longrightarrow{}$
\item Multiplikation (z.B. Karazuba-Multiplikation)
\end {itemize}
\includegraphics[scale=0.4]{karatsuba.jpg}
\end{frame}

\section {Exponentiation by squaring}
\begin{frame} {Naive Exponentiation} 
\lstinputlisting[language=c++]{exp.cpp}
Bei ICPC gehen wir davon aus, dass Multiplikation zweier Zahlen in $\mathcal{O}(1)$ liegt, also naive Exponentiation in $\mathcal{O}(n)$
\end{frame}

\begin{frame} {Idee}
Beobachtung:
\begin{equation}
   x^{n} =
   \begin{cases}
     (x^{2})^{n/2} & \text{f"ur n gerade} \\
      x*(x^{2})^{(n-1)/2} & \text{f"ur n ungerade} \\
   \end{cases}
\end{equation}
\end{frame}

\begin{frame} {Exponentiation by squaring, rekursive Implementierung}
\lstinputlisting[language=c++]{recExpSq.cpp}
\end{frame}

\begin{frame} {Exponentiation by squaring, iterative Implementierung}
\lstinputlisting[language=c++, basicstyle=\tiny]{iterExpSq.cpp}
Da Multiplikation konstant viel Zeit ben"otigt, liegt die Exponentiation$\mathcal{O}(log(n))$
\end{frame}


\begin{frame}{Hier kommt ein kleines Beispiel auf dem Tafel}
\end{frame}

\section{Kombinatorik}
\begin{frame}{Kombinatorik}
\begin{block}{Definition}
''Combinatorics is a branch of discrete mathematics concerning the study of countable discrete structures``\footnote{Competitive Programming 3}
\end{block}
\pause
Bei ICPC-Aufgaben erkennbar an:
\begin{itemize}
\item "Wie viele Moeglichkeiten gibt es, ..?"
\item "Berechne die Anzahl an X.."
\item Alles, was mit Zaehlen zu tun hat
\end{itemize}
\end{frame}

\begin{frame}{Der Mathetest}
\begin{block}{Aufgabe}
Lisa macht ein Austauschsemester in Australien. Um fuer einen Mathetest zu lernen, loest sie Rechen-Aufgaben, die ihr eine Kommilitonin diktiert hat. Leider hat die Kommilitonin nicht gesagt, wie die Aufgaben geklammert sind. \\
Gegeben die Anzahl an Faktoren, wie viele verschiedene Wege gibt es diese zu klammern?
\end{block}
\pause
Beispiel:
\begin{itemize}
\item Gegeben: $\lbrace a, b, c, d \rbrace$
\item Gesucht: Moeglichkeiten fuer Klammerung
\item $ a \left( b \left( c d \right) \right) $ , $\left( a b \right) \left( c d \right) $ , $\left( \left( a b \right) c \right) d$ , $\left( a \left( b  c \right) \right) d $ , $ a \left( \left( b  c \right) d \right) $
\end{itemize}
\end{frame}

\begin{frame}{Binomialkoeffizient}
Wie viele Moeglichkeiten gibt es, $k$ Objekte aus einer Menge von $n$ verschiedenen Objekten zu ziehen?
\pause
\begin{align*}
C\left( n,k \right) = \binom{n}{k} = \frac{n!}{\left( n-k \right) ! \times k!}
\end{align*}
\pause
Rekursive Definition:
\begin{align*}
C\left( n, 0 \right) &= C\left( n,n \right) = 1 \\
C\left( n,k \right) &= C\left( n-1,k-1 \right) + C\left( n-1,k \right)
\end{align*} %Mglkten fuer n-te Zahl plus k-1 Zahlen kleiner n + Mglkten fuer nicht n-te Zahl, also k-1 Zahlen kleiner n

\end{frame}

\begin{frame}{Binomialkoeffizient}
Tipps:
\begin{itemize}
\item Meist interessieren nicht alle Werte von $C \left( n, k \right)$
\begin{itemize}
\item Implementierung deshalb mit top-down
\end{itemize}
\item Fakultaet kann sehr gross werden
\begin{itemize}
\item benutze BigInteger
\item bei grossem $k$: $C \left( n, k \right) = C \left( n, n-k \right)$
\end{itemize}
\end{itemize}
\end{frame}

\begin{frame}{Catalan Numbers}
Definition:
\begin{align*}
Cat \left( n \right) &= \frac{1}{n+1} \binom{2n}{n} \\
&= \frac{\left( 2n \right)!}{\left( n+1 \right) \times n! \times n!} = \frac{\left( 2n \right) !}{\left( n+1 \right) ! \times n!} \\
Cat \left( n + 1 \right) &= \frac{\left( 2n + 2 \right) \times \left( 2n + 1 \right)}{\left( n + 2 \right) \times \left( n + 1 \right)} \times Cat \left( n \right)
\end{align*}
\end{frame}

\begin{frame}{Catalan Numbers}
$Cat \left( n \right)$ entspricht zum Beispiel:
\begin{itemize}
\item Anzahl verschiedener Binaer-Baeume mit $n$ Knoten
\item Anzahl korrekter Klammerausdruecke mit $n$ Klammerpaaren
\item Anzahl verschiedener Moeglichkeiten, $n+1$ Faktoren korrekt zu klammern
\end{itemize}
\end{frame}

\begin{frame}{Der Mathetest}
\begin{block}{Aufgabe}
Lisa macht ein Austauschsemester in Australien. Um fuer einen Mathetest zu lernen, loest sie Rechen-Aufgaben, die ihr eine Kommilitonin diktiert hat. Leider hat die Kommilitonin nicht gesagt, wie die Aufgaben geklammert sind. \\
Gegeben die Anzahl an Faktoren, wie viele verschiedene Wege gibt es diese zu klammern?
\end{block}
\pause
Loesung:
\begin{itemize}
\item Sei $n$ die Anzahl an Faktoren
\item $Cat \left( n-1 \right)$ loest die Aufgabe
\end{itemize}
\end{frame}

\begin{frame}{Zusammenfassung - Kombinatorik bei ICPC}
Die Loesung fuer eine Kombinatorik-ICPC-Aufgabe ist meist eine kurze rekursive Formel, oft in Verbindung mit Greedy oder DP. Der Aufwand liegt nicht in der Implementierung, sondern im Aufstellen der Formel.
\pause
\begin{itemize}
\item Kombinatorik-Aufgaben von \textbf{einer} Person bearbeiten lassen
\pause
\begin{itemize}
\item bestenfalls mit guten mathematischen Kenntnissen
\end{itemize}
\pause
\item Sobald die Formel fertig ist, Loesung coden und abgeben!
\end{itemize}
\end{frame}

\begin{frame}{Zusatz-Tipp!}
Gaenige Formeln sollte man kennen... \\

\end{frame}

\begin{frame}{Zusatz-Tipp!}
Gaenige Formeln sollte man kennen... \\



...oder ausprobieren! \\

\end{frame}

\begin{frame}{Zusatz-Tipp!}
\begin{block}{On-Line Encyclopedia of Integer Sequences}
Unter http://oeis.org/ kann man die ersten Loesungen fuer kleine Probleminstanzen eingeben und so pruefen, ob bereits eine Formel fuer diese Folge existiert.
\end{block}

\end{frame}


\section{Spieltheorie}
\begin{frame}{Spieltheorie allgemein}
\begin{itemize}
\item Nullsummenspiel
\pause
\item Genau ein Gewinner
\pause
\item Alle spielen perfekt
\pause
\item Gibt es eine Gewinnstrategie?
\end{itemize}
\end{frame}

\begin{frame}{Beispielspiel}
\begin{itemize}
\item sechs M"unzen, zwei Spieler
\pause
\item immer abwechselnd maximal drei M"unzen nehmen
\pause
\item Wer die letzte M"unze nimmt, gewinnt
\end{itemize}
\end{frame}

\begin{frame}{Spielbaum}
\begin{itemize}
\item Knoten: aktueller Spieler und Spielsituation
\pause
\item Kanten: legale Spielz"uge
\pause
\item Wurzel: Spielsituation beim Start
\pause
\item Bl"atter: Geben Bewertung (-1 oder 1) an
\pause
\end{itemize}
\end{frame}

\begin{frame}{Min-Max-Strategie}
\begin{itemize}
\item Min-Max-Strategie: Gewinn mit gr"o"stem Unterschied
\begin{equation}
   minmax(s,k) =
   \begin{cases}
     k.Bewertung & \text{für k Blatt-Knoten} \\
     min\left\{minmax(k') | k' Kindknoten\right\} & \text{falls s = min} \\
     max\left\{minmax(k') | k' Kindknoten\right\} & \text{falls s = max} \\
   \end{cases}
\end{equation}
\pause
\item Spieler-IDs k"onnen auch weggelassen werden: 
\begin{equation}
   minmax(k) =
   \begin{cases}
     k.Bewertung & \text{f"ur k Blatt-Knoten} \\
     - min\left\{minmax(k') | k' Kindknoten\right\} & \text{sonst} \\
   \end{cases}
\end{equation}
\end{itemize}
\end{frame}

\begin{frame}{Implementierung}
\begin{itemize}
\item wie gewohnt als Baum
\lstinputlisting[language=c++]{minmax1.cpp}
\pause
\item Spieler-IDs k"onnen weggelassen werden
\lstinputlisting[language=c++]{minmax2.cpp}
\end{itemize}
\end{frame}

\begin{frame}{manchmal Optimierung}
\begin{itemize}
\item Beispiel: Zahlen abwechselnd mit 2-9 multiplizieren
\pause
\item erster "uber n (Grenze) gewinnt
\pause
\item je acht Kinder: Viel zu viel
\pause
\item optimal: Immer abwechselnd 9 und 2
\end{itemize}
\end{frame}

\begin{frame}{Nim-Spiel}
\begin{itemize}
\item mehrere Haufen mit Objekten
\pause
\item zwei Spieler nehmen abwechselnd von einem Haufen
\pause
\item Wer das letzte Objekt nimmt, gewinnt
\pause
\item F"ur ein oder zwei Haufen mit Baum m"oglich 
\end{itemize}
\end{frame}

\begin{frame}{Nim-Spiel: Optimalstrategie}
\begin{itemize}
\item Anzahl Objekte in Haufen bin"ar mit xor verkn"upfen
\pause
\item Summe auf Null bringen, um zu gewinnen
\pause
\item Immer m"oglich
\end{itemize}
\end{frame}

\begin{frame}{Grundy-Zahlen}
\begin{itemize}
\item Theorem von Sprague-Grundy: Jedes neutrale Spiel "aquivalent zu Standard-Nim-Spiel
\pause
\item Grundy-Zahlen: kleinste Zahl, die nicht Grundy-Zahl von Nachfolgerstellung ist
\pause
\item Gewinnstrategie: Grundy-Zahl m"oglichst immer auf 0 bringen
\end{itemize}
\end{frame}


\section{Section 1}
\subsection{Subsection 1.1}
\begin{frame}{Example slide A}
\begin{itemize}
\item PCM, Citation: \cite{becker2008a} %\language
\pause
\item Bullet point 2
\item \dots
\end{itemize}
\end{frame}

\subsection{Subsection 1.2}
\begin{frame}{Example slide B}
\begin{block}{Block 1}
\begin{itemize}
\item Bullet point 1
\pause
\item Bullet point 2
\item \dots
\end{itemize}
\end{block}
\end{frame}

\section{Section 2}
\begin{frame}{Example slide C}
\begin{exampleblock}{Example 1}
\begin{itemize}
\item Bullet point 1
\pause
\item Bullet point 2
\item \dots
\end{itemize}
\end{exampleblock}
\end{frame}

\begin{frame}{Example slide D}
\begin{alertblock}{Alert 1}
\begin{itemize}
\item Bullet point 1
\pause
\item Bullet point 2
\item \dots
\end{itemize}
\end{alertblock}
\end{frame}

\appendix
\beginbackup

\begin{frame}[allowframebreaks]{References}
\printbibliography
\end{frame}

\backupend

\end{document}
